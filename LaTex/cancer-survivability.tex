\documentclass[review]{elsarticle}

\usepackage{lineno,hyperref}
\usepackage{float}
\usepackage{amsmath}
\usepackage[margin=1.5in]{geometry}
\usepackage{multirow}
\newcommand*\rot{\rotatebox{90}}
\usepackage{algorithm}
\usepackage{algpseudocode}

%\modulolinenumbers[5]
%\linenumbers

%\journal{Journal of \LaTeX\ Templates}

%%%%%%%%%%%%%%%%%%%%%%%
%% Elsevier bibliography styles
%%%%%%%%%%%%%%%%%%%%%%%
%% To change the style, put a % in front of the second line of the current style and
%% remove the % from the second line of the style you would like to use.
%%%%%%%%%%%%%%%%%%%%%%%

%% Numbered
%\bibliographystyle{model1-num-names}

%% Numbered without titles
%\bibliographystyle{model1a-num-names}

%% Harvard
%\bibliographystyle{model2-names.bst}\biboptions{authoryear}

%% Vancouver numbered
%\usepackage{numcompress}\bibliographystyle{model3-num-names}

%% Vancouver name/year
%\usepackage{numcompress}\bibliographystyle{model4-names}\biboptions{authoryear}

%% APA style
%\bibliographystyle{model5-names}\biboptions{authoryear}

%% AMA style
%\usepackage{numcompress}\bibliographystyle{model6-num-names}

%% `Elsevier LaTeX' style
\bibliographystyle{elsarticle-num}
%%%%%%%%%%%%%%%%%%%%%%%
\usepackage{lipsum}
\makeatletter
\def\ps@pprintTitle{%
	\let\@oddhead\@empty
	\let\@evenhead\@empty
	\def\@oddfoot{}%
	\let\@evenfoot\@oddfoot}
\makeatother

\begin{document}

\begin{frontmatter}
	
\title{Personalized colorectal cancer survivability prediction with machine learning methods }
%\author{Samuel Li \\ }
\author{Samuel Li \\ Princeton University, Princeton, NJ \\ \href{mailto:seli@princeton.edu}{seli@princeton.edu}} 

%\title{Elsevier \LaTeX\ template\tnoteref{mytitlenote}}
%\tnotetext[mytitlenote]{Fully documented templates are available in the elsarticle package on \href{http://www.ctan.org/tex-archive/macros/latex/contrib/elsarticle}{CTAN}.}

%% Group authors per affiliation:
%\author{Elsevier\fnref{myfootnote}}
%\address{Radarweg 29, Amsterdam}
%\fntext[myfootnote]{Since 1880.}

%% or include affiliations in footnotes:
%\author[mymainaddress,mysecondaryaddress]{Elsevier Inc}
%\ead[url]{www.elsevier.com}

%\author[mysecondaryaddress]{Global Customer Service\corref{mycorrespondingauthor}}
%\cortext[mycorrespondingauthor]{Corresponding author}
%\ead{support@elsevier.com}

%\address[mymainaddress]{1600 John F Kennedy Boulevard, Philadelphia}
%\address[mysecondaryaddress]{360 Park Avenue South, New York}

\begin{abstract}
In this work we investigate the importance of ethnicity in colorectal cancer survivability prediction using machine learning techniques and the SEER cancer incidence database. We compare model performances for 2-year survivability prediction and factor importance rankings between Hispanic, white, and mixed patient populations. Our models consistently performed better on single-ethnicity populations, and provided different factor importance rankings when trained on different populations. Additionally, after training our models to achieve higher AUC score than the best reported in literature, we improved G-mean score with imbalanced classification techniques. These results provide evidence in favor for increased consideration of patient ethnicity in cancer survivability prediction, and for more personalized medicine in general.
\end{abstract}

\begin{keyword}
Cancer survivability prediction \sep SEER \sep machine learning \sep personalized medicine \sep imbalanced classification
%\texttt{elsarticle.cls}\sep \LaTeX\sep Elsevier \sep template
%\MSC[2010] 00-01\sep  99-00
\end{keyword}

\end{frontmatter}

\section{Introduction}
Colorectal cancer, defined as cancer starting in the colon or the rectum, is among the most common types of cancer for both men and women. The probability of developing colon or rectum cancer is about 4.5\% for men and 4.15\% for women. In 2018, the National Cancer Institute estimates there will be 140,250 new diagnoses of colorectal cancer and over 50,630 deaths caused by these cancers in the United States\cite{colorectal-stats}. This makes colorectal cancer the third leading cause of cancer-related deaths in the country. A patient’s survival time is largely dependent on the state of their cancer at the time of diagnosis: the 5-year survival rate for people with stage I colon cancer is approximately 92\%, while the 5-year survival rate of stage IV colon cancers is approximately 11\%\cite{colorectal-survival-stats}.

Upon receiving their diagnosis, a patient will likely want to know their x-year survivability, defined as the probability that they will survive beyond x years. An estimate of their survivability based only on stage of cancer does not account for all the personal factors specific to that patient, and will be far less accurate than a prediction model that takes into account factors like patient age, race, primary site, etc. While it is difficult for a doctor to give consideration to many factors, machine learning algorithms can efficiently find patterns in large datasets of patient data. Given the vast amount of diagnostic data that is available and continuously being generated, there are many opportunities to use machine learning techniques to provide more personalized and accurate survivability predictions. 

There is a sizable literature dedicated to the application of machine learning in cancer prognosis and survivability prediction. Delen et al. used data mining methods to predict breast cancer survivability in 2004\cite{delen2005predicting}. Since then, with the increased adoption of machine learning methods, researchers have applied a wide variety of algorithms to survivability prediction of many types of cancer. Studies include use of deep learning\cite{delen2005predicting,al2017survivability}, ensemble methods\cite{al2013colon}, and imbalanced classification techniques\cite{liu2009decision}. These studies all predict survivability based on detailed personal and diagnostic information for each patient.

One approach to provide patients with more personalized healthcare is to take the patient's race into account. It is well known that people of different racial groups have varying levels of susceptibility and responses to different diseases. Causes for discrepancies may include anything from genetic differences to environmental influences to cultural factors. Currently, there is significant disparity in colorectal incidence and mortality rates between different racial groups. From 2011 to 2015, the National Cancer Institute reported that the colorectal cancer incidence rate was 38.8 per 100,000 for the white population, while the Hispanic and black populations had incidence rates of 47.6 and 33.5, respectively. In the same years, the white, Hispanic, and black populations had recorded mortality rates of 14.1, 11.5, and 19.4, respectively\cite{race-mortality}.

The aim of our research is to develop models which effectively predict colorectal cancer survivability, and to use these models to compare factor importances between populations of different race. In this paper, we apply machine learning methods to predict 2-year survivability for Hispanic and white colorectal cancer patients, and compare factor importances in determining survivability. These two groups were chosen since they were the most frequently appearing ethnic groups in our dataset. While “Hispanic” describes an ethnicity, and therefore the two groups are not automatically disjoint by official race and ethnicity records, most self-described Hispanic people consider their Hispanic background as part of their racial background\cite{hispanic_origin}. In this study, we consider a patient white if their race encoding is white and their Hispanic origin encoding labels them as non-Hispanic.

Additionally, we investigate how we can improve upon the problem as an imbalanced classification problem. An imbalanced classification problem is defined as a problem in which the instance labels, in our case whether or not a patient survives beyond two years, are disproportionate. Table~\ref{dataset-info} shows that the labels are disproportionate in an approximately 4:1 survived versus not-survived ratio. We apply established imbalanced classification techniques to improve imbalanced classification performance.

We created a separate dataset for each race and trained classifiers to predict survivability on the separated data. The trained models provide importance rankings of the patient factors (age at diagnosis, tumor size, primary site, etc.), which allow factor comparison between Hispanic and white patients. We also apply imbalanced classification techniques to improve the performance of our models. 

This paper is organized as follows: Section 2 provides background information about the dataset, data processing procedures, models used, and performance metrics. Section 3 details our methodology. Results are reported in Section 4, and we discuss our findings in Section 5 before concluding with Section 6.
\section{Background and Methodology}

\subsection*{Machine Learning Pipeline}
Our machine learning pipeline consist of the distinct stages of preprocessing, model training, and evaluation. Preprocessing is everything that is done to convert raw data into a format that can be used as input to our models. The models that we select are trained on the cleaned data, and then they are evaluated by a variety of metrics. The following sections provide further details on each stage.
\subsection*{Data Source}
The diagnostic data used in our research was obtained from the Surveillance, Epidemiology, and End Results (SEER) program database\cite{seer-dataset}. Sponsored by the National Cancer Institute, SEER is the principal repository for cancer incidence and survival data in the United States. The SEER database contains over 10 million diagnostic records from 1973-2015. These records are collected from cancer registries across the country, and include data from over ten types of cancer. We aggregate patient data from all geographical registries, and filtered our data to only include white and Hispanic patients. The mixed dataset is simply formed by combining the Hispanic and white datasets. Table~\ref{dataset-info} shows the survivability distribution by ethnicity.

\begin{table}[H]
	\centering
	\caption{Dataset label distribution by ethnicity}
	\begin{tabular}{cccc}
		\textbf{Ethnicity} & \textbf{Not Survived} & \textbf{Survived} & \textbf{Total} \\ 
		\hline
		White              & 53343 (19.4\%)   & 222204 (80.6\%)        &  275547  \\
		Hispanic           & 6762 (18\%)       & 30813 (82\%)           &  37575 \\
	\end{tabular}
	
	\label{dataset-info}
\end{table}

\subsection*{Data Preprocessing}
In this study, we used SEER data from the years 2004-2015, for a couple of reasons. There was a major variable recode in the data beginning in 2004 which may have compromised data consistency if we had used patient data both prior and after 2004. Additionally, using relatively recent data is ideal because medical technology is continuously developing and improving, resulting in more timely, accurate diagnoses and better treatments. We used colorectal cancer patient data from all available registries.

Raw SEER data is encoded in pure ASCII files. These files were converted via Python scripts to a tabular format, from which we created two distinct datasets: one containing only Hispanic patients, the other containing only white patients. This separation is done prior to normalization and imputation so that those modifications are specific to each race.

In this paper, we will refer to instances and predictions where a patient survives beyond two years as “positive”, and instances and predictions where a patient dies of colorectal cancer before two years as “negative”. Accordingly, an instance is labeled negative if the survival time in months field is less than 24, and the cause of death is colon or rectum cancer. All instances not satisfying those two conditions are labeled positive.

Of the 133 features provided by SEER, we selected the 20 features detailed in Table~\ref{variable-info} based on feature importances reported in the literature\cite{delen2005predicting,al2013colon}. We selected 16 categorical features and 4 continuous features. Each feature was carefully examined for outliers, missing values, and other inconsistencies. For the age diagnosed, number of malignant tumors, and months survived features, we deleted entire instances if the feature was missing. For the tumor size and number of positive nodes features,  where there were a large proportion of missing or unknown values, we employed the MICE imputation technique, described shortly. It is worth noting that many other related studies cited in this paper have dropped large proportions of instances with missing data. However, we wanted to minimize instance eliminations because the SEER dataset did not have abundantly many Hispanic patient instances to begin with.

\begin{table}[H]
	\centering
	\caption{Features used in models}
	\tiny
	\begin{tabular}{ll}
		\textbf{Categorical Variable} & \textbf{Unique values}  \\ 
		\hline
		Marital status    & 7                       \\
		Sex               & 2                       \\
		Primary site      & 13                      \\
		Histology         & 139                     \\
		Behavior          & 2                       \\
		Grade             & 5                       \\
		Diagnostic confirmation  & 8                       \\
		Extension         & 65                      \\
		Lymph Nodes       & 18                      \\
		Metastasis        & 25                      \\
		Tumor size evaluation  & 7                       \\
		Node evaluation        & 7                       \\
		Metastasis evaluation  & 7                       \\
		Surgery site      & 34                      \\
		Reason no surgery & 8                       \\
		Summary stage     & 5                       \\
		Registry          & 18                      \\
	\end{tabular}
	\quad
	\centering
	\begin{tabular}{lll}
		\textbf{Continuous Variable} & \textbf{Mean} & \textbf{S.D.}  \\ 
		\hline
		Age diagnosed     & 68.3          & 14             \\
		Positive nodes    & 1.57          & 4.26           \\
		Number of tumors  & 1.4           & 0.717          \\
		Tumor size        & 43.1          & 37.3          
	\end{tabular}
	
	\label{variable-info}
\end{table}

In order to reconcile categorical and continuous features, we used a one-hot encoding scheme to binarize the categorical features. For each categorical feature, one-hot encoding creates a new binary indicator variable for every possible value of the original feature. 

\subsection*{MICE Imputation}
MICE is among the most common strategies for handling missing values in electronic health data\cite{azur2011multiple}. MICE initially fills all missing values with the mean or median of the corresponding non-missing features, and then iteratively cycles through each feature requiring imputations and performs regression by treating all other features as independent variables. Algorithm \ref{mice-algo} provides pseudocode for the MICE algorithm.
	
\begin{algorithm}[H]
	\hspace*{\algorithmicindent} \textbf{Data:} input data; \\ \hspace*{\algorithmicindent} \textbf{numcycles:} number of cycles to run MICE (usually 10)
	\begin{algorithmic}[1]
		\Function{MICE}{Data, numcycles}
			\State Perform mean/median imputation on all missing values in dataset
			\For{i in numcycles}
				\For {feature $f_m$ in set of features with any missing values}
					\State Regression(Data - $column_{f_m}$, $column_{f_m}$)
					\State Replace originally missing features in $column_{f_m}$ with regression result
				\EndFor
			\EndFor
		\EndFunction
		
	\end{algorithmic}
	\caption{MICE Imputation}
	\label{mice-algo}
\end{algorithm}



\subsection*{Models}
We selected the following classifiers to predict survivability. Models were selected based on preliminary test performance as well as models previously reported in literature.
\begin{itemize}
	\item \textit{Logistic Regression} Logistic Regression is a classifier that evaluates the weighted sum of the input components, and applies the sigmoid function to the weighted sum. The output from the sigmoid function can be interpreted as the probability of the positive class in the binary classification scenario. The weights are learned during training.
	\item \textit{Random Forest} A Random Forest is an ensemble of decision trees, where each tree splits nodes based on random features (instead of best feature)\cite{breiman2001random}. The final output class is the class which received the majority vote over the individual trees. Randomly determining which features to split on reduces overfitting because there is no guarantee that a certain feature will be included in each model. 
	\item \textit{AdaBoost} AdaBoost\cite{freund1999short} is an ensemble of sequentially trained classifiers. Each instance in the training set is initialized with equal weight. Misclassified instances are given increased weight while training the next classifier. Prediction consists of a weighted prediction on all the classifiers, where each is weighted by its accuracy during training. In this study, the decision stump is used as the base classifier.
	\item \textit {Neural Network} A neural network is composed of layers of nodes with interconnected weights. Each node receives a weighted sum of the outputs from all the nodes in the previous layer. An activation function is applied to the weighted sum, and the result is the output of the node. The weights in each layer are determined by the backpropagation training method, which updates each weight based on its influence on the loss function. Neural networks are at the forefront of recent deep learning progress\cite{lecun2015deep}.
	\item \textit {Imbalanced Classification} We have additionally employed the imbalanced classification techniques of weighted classes and undersampling to improve the performance of the above models. Weighted classes penalizes the model more harshly for mislabeling an instance of the minority class. Undersampling only uses a fraction of the majority class instances, to make the proportion between class instances more balanced\cite{he2008learning}.
	
	%We have also attempted oversampling and SMOTE\cite{chawla2002smote}, but found that those methods yielded inferior results.
	
\end{itemize}

\subsection*{Evaluation Metrics}
As our class labels are imbalanced, the conventional metric of accuracy is an inadequate indicator of model performance: A model which always predicts that the patient survives would achieve over 80\% accuracy. The most common evaluation metric in the literature is Area Under Curve (AUC) of the ROC curve. The AUC score is intuitively interpreted as the probability that, given a positive and a negative instance, the positive instance is ranked more likely to be positive than the negative instance. The AUC score can be calculated given a set of predicted labels and the corresponding true labels. We employ AUC as the primary evaluation metric so that our results can be more comparable to results reported in the literature.

Additionally, we use the G-mean as an indicator for our performance on an imbalanced classification problem. To define the G-mean, we first define the following four metrics:
\begin{align*}
\text{TP (true positive): the number of positive instances labeled positive}\\
\text{TN (true negative): the number of negative instances labeled negative}\\
\text{FP (false positive): the number of negative instances labeled positive}\\
\text{FN (false negative): the number of negative instances labeled negative}
\end{align*} 
The confusion matrix shown in Table~\ref{confusion-matrix} summarizes the definitions.
\begin{table}[H]
	
	\caption{The confusion matrix groups instances by the combination of their true and predicted labels.}
	\centering
	\begin{tabular}{llll}
		&                                   & \multicolumn{2}{c}{True Label}                     \\ 
		\cline{3-4}
		&                                   & Not Survived            & Survived                 \\ 
		\cline{3-4}
		\multirow{2}{*}{\rot{Predicted}} & \multicolumn{1}{|l|}{Not Survived} & \multicolumn{1}{l|}{TN} & \multicolumn{1}{l|}{FN}  \\ 
		\cline{3-4}
		\multicolumn{1}{l}{}                & \multicolumn{1}{|l|}{Survived}     & \multicolumn{1}{l|}{FP} & \multicolumn{1}{l|}{TP}  \\ 
		\cline{3-4}
		&                                   &                         &                         
	\end{tabular}
	
	\label{confusion-matrix}
\end{table}
\noindent The G-mean is defined as follows:
\begin{align*}
	\text{G-mean} = \sqrt{\text{sensitivity}\cdot\text{specificity}}
\end{align*}
where sensitivity and specificity are defined as:
\begin{align*}
\text{sensitivity} = \frac{TP}{TP + FN} \quad\text{specificity} = \frac{TN}{TN + FP}
\end{align*}
 The G-mean is a viable metric for imbalanced classification because it weights the rate of correctly predicting the majority and minority classes equally.
\section{Results}
We trained our models with Scikit-learn, an open-source Python machine learning library. We used Scikit-learn version 0.19.1 and Python 3.6.4. All experiments were executed on a Linux machine with an Intel i7-3700 3.4 GHz processor and 16GB RAM running Ubuntu 18.04 LTS.

We used 5-fold cross-validation to determine the best hyperparameters for each model, and recorded results on the test sets, each composed of a randomly-selected 10 percent of the data. All code used is in our GitHub repository\cite{cancer_survivability}. Table~\ref{auc-scores} details the AUC scores for each model and dataset. The AUC scores we achieved with our neural networks are above the best reported in literature of 0.8675 by\cite{al2017survivability}, which was also produced by neural networks. Our AUC scores are calculated on each model's prediction of the probability that the true label is positive. This will provide a higher AUC score because wrong predictions are not penalized as heavily if their probability prediction was closer to 0.5.
\begin{table}[H]
	\centering
	\caption{AUC scores for trained models on the test set. The highest reported AUC for 2-year colorectal cancer survivability had been 0.8675\cite{al2017survivability}. Our neural network architecture had 3 fully-connected hidden layers of 400 neurons, ReLU activation, 0.1 dropout between each layer besides input layer, with sigmoid output activation.}
	\begin{tabular}{llll}
		\textbf{Model}      & \textbf{Hispanic} & \textbf{White} & \textbf{Mixed}  \\ 
		\hline
		Logistic Regression & .859              & .872           & .870             \\
		Random Forest       & .855              & .865           & .849            \\
		AdaBoost            & .859              & .871           & .859            \\
		Neural Network       & .873              & .875           & .856           
	\end{tabular}
	
	\label{auc-scores}
\end{table}

Table~\ref{factor-rankings} ranks the top seven factors for each model and dataset. Factor models were obtained from the coefficient and parameter values of the trained models, in the context of each model. In the logistic regression model, a larger magnitude coefficient for a feature indicates that the feature has relatively greater impact on the label prediction. For decision tree based models like random forest and AdaBoost, we use Gini impurity and entropy scores for each feature, which reveal the feature that provide the most information when a decision tree node is split on that feature. Since our models are trained on one-hot encoded features, we sum the feature importance values for all the one-hot encoded columns corresponding to a single original feature (histology, surgery site, etc.) to obtain the score used to rank feature importances. Scikit-learn allows access to model parameters, and also includes class functions which calculate factor rankings.


\begin{table}[H]
	\centering
	\footnotesize
	\caption{Factor rankings for each model and ethnic population}
	\begin{tabular}{ll|ll|ll}
		\multicolumn{2}{c}{\textbf{Logistic Regression}}         & \multicolumn{2}{c}{\textbf{Random Forest}}              & \multicolumn{2}{c}{\textbf{AdaBoost}}                   \\
		\multicolumn{1}{l}{\textbf{Hispanic}} & \textbf{White}   & \multicolumn{1}{l}{\textbf{Hispanic}} & \textbf{White}  & \multicolumn{1}{l}{\textbf{Hispanic}} & \textbf{White}  \\ \hline
		Histology                             & Histology        & Metastasis                            & Metastasis      & Extension                             & Extension       \\
		Extension                             & Lymph node inv.  & Stage                                 & Stage           & Histology                             & Age             \\
		Metastasis                            & Extension        & Age                                   & Age             & Age                                   & Histology        \\
		Surgery site                          & Surgery site     & No surg. reason                       & No surg. reason & Tumor size                            & Positive nodes  \\
		Diagnostic conf.                      & Metastasis       & Positive nodes                        & Positive nodes   & Positive nodes                        & Metastasis      \\
		Lymph node inv.                       & Diagnostic conf. & Surgery site                          & Surgery site    & Metastasis                            & Tumor size      \\
		No surg. reason                       & Primary site     & Tumor size                            & Tumor size      & Surgery site                          & Surgery site   
	\end{tabular}
	
	\label{factor-rankings}
	
\end{table}

Table~\ref{gmean-compare} shows the improvements made by using the weighted classes and undersampling imbalanced classification methods. These methods were applied to the datasets prior to model training. Our results are comparable to the G-mean score of 0.792 reported by Al-Bahrani and Agrawal\cite{al2017survivability}.
\begin{table}[H]
	\caption{G-mean scores for selected models and ethnic populations. We undersampled until we had an equal number of both labels, and we weighted the minority class five times as the majority class}
	\centering
	\begin{tabular}{llll}
		\textbf{Model}               & \textbf{Hispanic} & \textbf{White} &   \\ \hline
		Logistic Regression          & .628              & .683           &   \\
		Weighted Logistic Regression & .783              & .800             &   \\
		Undersampled Logistic Regression & .790 & .800 & \\ 
		Random Forest                & .623              & .631           &   \\
		Weighted Random Forest       & .782              & .775           &  \\
		Undersampled Random Forest & .787 & .796
		
	\end{tabular}
	
	
	\label{gmean-compare}
\end{table}

\section{Discussion}
Our results provide insight into the importance of patient ethnicity in cancer survivability prediction, as well as the cancer survivability prediction problem itself. From the AUC score table~\ref{auc-scores}, we observed that models trained on mixed-ethnicity patient data performed consistently worse than those trained on single-ethnicity data. Logistic regression was the only model in which the mixed dataset yielded better results either the Hispanic or white datasets. The white dataset had significantly better results than the mixed dataset for all models, while the Hispanic dataset had slightly better results. This is very likely due to the Hispanic dataset being far smaller than the white dataset, rather than the Hispanic dataset being intrinsically harder to train on. 

Our factor rankings results show fairly consistent variable rankings between the Hispanic and white datasets for the same model, with minor differences. For example, lymph node involvement is the second most important factor for white patient instances in the logistic regression model, but the sixth most important factor for Hispanic patient instances. It is expected that there are not dramatic differences in factor rankings between the two groups. However, even minor differences can be of use in the analysis of a patient’s diagnostic data; for example, by giving an observation of a feature more consideration for a patient with a certain race. Additionally, we observe in table~\ref{factor-rankings} that different models have widely varying factor rankings. This confirms that many factors are not independent, and that there are many viable ways to make survivability predictions with competitive results. We note that neural networks are unable to provide factor rankings due to the non-transparent nature of the model.

From the imbalanced classification results, we observe that the logistic regression and random forest models both perform better with respect to the G-mean metric with imbalanced classification techniques. Both weighted classes and undersampling improved the G-mean significantly. Applying these techniques to the AdaBoost and neural network model achieves similar results(appendix Table~\ref{more-scores}). The same table shows that G-mean scores greatly improved, while AUC scores slightly increased after applying undersampling to AdaBoost, and slightly decreased after applying undersampling to a neural network. Applying imbalanced classification methods demonstrates the trade-off between sensitivity and specificity, and the inherent hardness of this classification problem. In the original models sensitivity and specificity values were around 0.9 and 0.4, respectively. After applying either weighted classes or undersampling, sensitivity and specificity values were both close to 0.8. 

The trade-off implies that instances with similar feature values have different labels. In some instances this is due to both actual survival times being close in value but on opposite sides of the 24-month cutoff. In other instances, survival times can actually differ widely, suggesting that we do not have all the relevant information. Indeed, this is the case: we only have the colorectal cancer diagnostic information and basic personal information. We hypothesize that more extensive data which includes factors like personal health history and lifestyle information would improve performance beyond the best currently reported in literature. However, there is currently no dataset with such degree of patient information.


\section{Conclusion}
In this study we applied machine learning methods to predict 2-year colorectal cancer survivability. Specifically, we trained logistic regression, random forest, AdaBoost, and neural network models on SEER data we carefully preprocessed. Our models achieved AUC score of 0.87, improving upon the best-reported score in the literature [REF]. We used our models to investigate the impact of ethnicity on model performance, as well as diagnostic factor rankings. Our models achieved superior performance on single-ethnicity patient data compared to the mixed data, and we also observed minor differences in factor importance. Finally, we used the imbalanced classification methods of weighted classes and undersampling to improve the G-mean score to approximately 0.8. 

We infer from our results that cancer survivability prediction is a difficult classification problem, and that the SEER dataset may not contain enough information to significantly improve the performance of our current models. Future work includes extending our focus on ethnicity beyond the Hispanic and white patient groups, as well as to other types of cancer.


\section*{References}

\bibliography{mybibfile}

\begin{table}[H]
	\centering
	\begin{tabular}{lllll}
		& \multicolumn{2}{c}{\textbf{Hispanic }} & \multicolumn{2}{c}{\textbf{White}}  \\
		\textbf{Model}              & \textbf{G-mean} & \textbf{AUC}         & \textbf{G-mean} & \textbf{AUC}      \\ \hline
		AdaBoost                    & 0.627           & 0.859                & 0.671           & 0.859             \\
		Undersampled AdaBoost       & 0.787           & 0.865                & 0.798           & 0.871             \\
		Neural Network              & 0.6             & 0.873                & 0.7             & 0.875             \\
		Undersampled Neural Network & 0.788           & 0.868                & 0.796           & 0.873             \\
		&                 &                      &                 &                  
	\end{tabular}
	\caption{G-mean scores of AdaBoost and neural networks, as well as G-mean and AUC comparisons for imbalanced classification.}
	\label{more-scores}
\end{table}

\end{document}